\documentclass{article}
\providecommand{\RevisionInfo}{}
\usepackage{hyperref}
% Reformat section titles
\usepackage{titlesec}

% This package is useful for debugging label problems
% Comment out in final revision
%\usepackage{showkeys}

% Title page information
\title{UNO Open Works Collective}
\author{Various}
% Last Modified Date
\newcommand{\datechanged}{Last Updated: \RevisionInfo}
\date{\datechanged}

% Fix margins
\setlength{\evensidemargin}{0in}
\setlength{\oddsidemargin}{0in}
\setlength{\textwidth}{6.5in}
\setlength{\topmargin}{0in}
\setlength{\textheight}{8.5in}

% Use \article for articles and \asection for sections of articles.
% Automatically provide labels with the same article or section title.
\newcommand{\article}[1]{\section{#1} \label{#1}}
\newcommand{\asection}[1]{\subsection{#1} \label{#1}}
\newcommand{\asubsection}[1]{\subsubsection{#1} \label{#1}}
\renewcommand{\thesection}{\Roman{section}}
\renewcommand{\thesubsection}{\arabic{section}.\Alph{subsection}}
\renewcommand{\thesubsubsection}{\arabic{section}.\Alph{subsection}.\arabic{subsubsection}}
\titleformat{\section}{\normalfont\Large\bfseries}{Article \thesection}{1em}{}
\titleformat{\subsection}{\normalfont\large\bfseries}{Section \thesubsection}{1em}{}

% Adding an \asubsubsection -- I feel dirty
%\setcounter{secnumdepth}{5}
\newcommand{\asubsubsection}[1]{\paragraph{#1} \label{#1}}
\renewcommand{\theparagraph}{\arabic{section}.\Alph{subsection}.\arabic{subsubsection}.\Alph{paragraph}}

% Adding \a(sub){3,4}section during merge of bylaws and articles -- I feel _really_ dirty
\setcounter{secnumdepth}{7}
\setcounter{tocdepth}{7}
\newcommand{\asubsubsubsection}[1]{\parindent=0em\subparagraph{#1} \label{#1}}
\renewcommand{\thesubparagraph}{\arabic{section}.\Alph{subsection}.\arabic{subsubsection}.\Alph{paragraph}.\arabic{subparagraph}}

\newcounter{asubsubsubsubsection}[subparagraph]
\renewcommand{\theasubsubsubsubsection}{\arabic{section}.\Alph{subsection}.\arabic{subsubsection}.\Alph{paragraph}.\arabic{subparagraph}.\Alph{asubsubsubsubsection}}
\newcommand{\asubsubsubsubsection}[1]{\parindent=0em\refstepcounter{asubsubsubsubsection}\par\textbf{\theasubsubsubsubsection\hspace{1em}#1 \label{#1}}}

% Headings
\pagestyle{myheadings}
\markright{{\rm CSH Constitution \hfill \datechanged \hfill Page }}

% Reference example:
%Test reference \ref{House Objectives} House Objectives.


\begin{document}
% Title
\maketitle

\tableofcontents

% ARTICLE I Introduction
\newpage
\article{Introduction}

\asection{Name}
The name of this organization shall be the: 
Open Works Collective;
Abbreviated: OWC

\asection{Purpose}
The purpose of this organization is to garden a productive open ecosystem for communities associated with the University of New Orleans to foster interdisciplinary creativity through the intersection of inclusivity, transparency, and collaboration.

\asection{Objectives}
The objectives of the Open Works Collective are:
\begin{enumerate}
	\item To organize/promote open forms of collaboration with a focus in our university community
	\item To root ourselves in diversity of perspective, thought and interest
	\item To foster mutualistic communitcation across disciplines and departments 
	\item To support the rigorous exploration of creative projects and research
	\item To maintain a common space, digital and otherwise, that is inclusive and accepting to all people
	\item - This space should facilitate community interconnectivity/function/health/growth and be resilient to harmful speech or actions
	\item To develop and offer resources for education relevant to our purpose 
\end{enumerate}

% IDEALS: A potential section regarding our influencing ideals and where they come from - NEEDS REVIEW FROM CEDRIC & RAVEN
\asection{Ideals}
Some foundational ideals of the Open Works Collective are:
\begin{enumerate}
	\item Openness as derived from the open movement - for the goal of reducing friction in the sharing of information  
	\item Inclusivity for all 
	\item Ineropable design - structured for 
	\item To provide a friendly and comfortable living environment in the residence halls
\end{enumerate}

%
% ARTICLE II - MEMBERSHIP
%
\newpage
\article{Membership}
There are five major types of membership available to Computer Science House.
Each carries different qualifications, expectations, and privileges.
When describing the different memberships available, the following terms are used:
\begin{description}
	\item[Qualifications:] What qualifications an applicant needs to apply
	\item[Selection:] The process by which an applicant gains membership
	\item[Expectations:] The duties and responsibilities of House members
	\item[Privileges:] The benefits offered to House members
	\item[Evaluations:] The process by which a member is reviewed and assessed
	\item[Leave of Absence:] The process by which a member may leave House for a period of time
	\item[Resignations:] The process by which a member terminates House membership
	\item[Term:] The length of time the membership lasts
\end{description}

\asection{Community Membership}
\asubsection{Community Membership Qualifications}
Introductory Membership is open to all students at the Rochester Institute of Technology.
\asubsection{Community Membership Selection}
Applicants notify the House of their interest in membership by submitting an application to the Evaluations Director.
They must then undergo the selection process as defined in \ref{Selection Processes}.
\asubsection{Community Membership Expectations}
Introductory members are expected to meet all the requirements of the Introductory Process, as described in \ref{Expectations of an Introductory Member}.
\asubsection{Community Membership Privileges}
Introductory members receive the right to use Computer Science House facilities, to attend House functions, and to have housing priority over all persons who are not Computer Science House members.
\asubsection{Community Membership Evaluations}
Introductory members are evaluated on their performance during the introductory period.
The introductory evaluation process is described in \ref{Introductory Evaluation}.
\asubsection{Community Membership Leave of Absence}
After completion of the Introductory Packet, an Introductory member may request a leave of absence through the process described in \ref{Leave of Absence}.
\asubsection{Community Membership Resignations}
Introductory members may resign by submitting the request for termination of membership to the Chairperson, or Evaluations Director in writing before the completion of the introductory process.
\asubsection{Community Membership Term}
Introductory membership shall last until the end of the introductory process, at which time active membership is granted or membership is revoked.

\asection{Student Organization Membership}
\asubsection{Community Membership Qualifications}
Introductory Membership is open to all students at the Rochester Institute of Technology.
\asubsection{Community Membership Selection}
Applicants notify the House of their interest in membership by submitting an application to the Evaluations Director.
They must then undergo the selection process as defined in \ref{Selection Processes}.
\asubsection{Community Membership Expectations}
Introductory members are expected to meet all the requirements of the Introductory Process, as described in \ref{Expectations of an Introductory Member}.
\asubsection{Community Membership Privileges}
Introductory members receive the right to use Computer Science House facilities, to attend House functions, and to have housing priority over all persons who are not Computer Science House members.
\asubsection{Community Membership Evaluations}
Introductory members are evaluated on their performance during the introductory period.
The introductory evaluation process is described in \ref{Introductory Evaluation}.
\asubsection{Community Membership Leave of Absence}
After completion of the Introductory Packet, an Introductory member may request a leave of absence through the process described in \ref{Leave of Absence}.
\asubsection{Community Membership Resignations}
Introductory members may resign by submitting the request for termination of membership to the Chairperson, or Evaluations Director in writing before the completion of the introductory process.
\asubsection{Community Membership Term}
Introductory membership shall last until the end of the introductory process, at which time active membership is granted or membership is revoked.

\asection{Steering Comittee Membership}


\asection{Advisor Membership}

\asection{Mentor Membership}


\newpage
%
% ARTICLE III - VOTING
%
\article{Voting}
This section outlines how Open Works Collective makes its decisions and defines relevant terminology.
\asection{Definitions}
\asubsection{Total Number of Possible Votes}
The sum of the number of Active members eligible to vote.
\asubsection{Total Number of Votes Cast}
The Total Number of Votes Cast is defined as the total number of votes received for every voting option minus the number of abstentions.
\asubsection{Quorum}
A Quorum is defined by the minimum number of votes cast required for a vote to be official.
It is a fraction or a percentage of the total number of possible votes.
Any member present for an Immediate Vote or given a ballot who does not explicitly cast their vote is counted as an Abstention.
A Quorum is reached if the Total Number of Votes Cast plus the number of Abstentions is equal to or exceeding the minimum number of votes required.
\asubsection{Proxy Ballot}
A Proxy Vote is defined as any ballot that was cast by one member on behalf of another member.
Any member may cast a Proxy Vote for another member who is unable to actually participate in the vote.
A Proxy Vote must be explicitly written down and signed by the member not in attendance.
The count of all Proxy Vote must be recorded and announced in all votes.
Proxy Votes are only permissible where explicitly stated, at the discretion of the Chair of the Vote.
\asubsection{Abstention}
An Abstention is defined as a vote indicating a neutral position in the vote.
A means to abstain must always be provided in a vote.
Abstentions are counted towards a Quorum, but not towards the Total Number of Votes Cast used to determine if a vote passes or not.
\asubsection{Vote Counters}
The current President of the Open Works Collective is a vote counter and will additionally select two other members to count votes.

\asection{Types of Voting}
\asubsection{Scheduled Vote}
\asubsubsection{Method of Vote}
Votes are cast on paper ballots, which provide a means to indicate every possible option in the vote.
A ballot is then distributed to each acitve Open Works Collective member that is eligible to cast a vote and then votes are collected in the designated ballot box for a pre-specified length of time.
At the end of the voting period, the Chair of the Vote collects the ballots, closing the voting period.
The Vote Counters then tally the results.
\asubsubsection{Voting Period}
For constitutional modification, candidate selection, and officer removal votes, the voting period must be at least forty-eight (48) hours in length.
For any other type of vote, the voting period must be at least twenty-four (24) hours.
The minimum length of the voting period may be explicitly lengthened, but never shortened, in the text describing the actual vote.
\asubsection{Instantaneous Vote}
\asubsubsection{Method of Vote}
The current President of the Open Works Collective will state all possible ways to vote, then call out each possibility one at a time.
The chairing member will count the number of members casting their instantaneous vote for that possibility. Instantaneous votes will be planned in advanced and will be held during meetings. The current President of the Open Works Collective will decided whether or not the subject being voted on should be instantaneous or scheduled.
\asubsubsection{Voting Period}
An instantaneous vote lasts as long as it takes for all votes to be tallied.

\asection{Number of Votes Required}
The Number of Votes Required refers to the numbers required to achieve a quorum and for a vote to pass.
Below are listed four standard votes.
Numbers for non-standard votes are defined in the section describing the actual vote.
\asubsection{Simple Majority}
In a Simple Majority Vote, a Quorum is reached if the Total Number of Votes Cast is equal to or exceeds one-half the Total Number of Possible Votes.
An option in the vote passes if the number of votes cast for that option is larger than the number of votes cast for every other option individually.
\asubsection{Fifty Percent}
In a Fifty Percent Vote, a Quorum is reached if the Total Number of Votes Cast is equal to or exceeds one-half the Total Number of Possible Votes.
An option in the vote passes if the number of votes cast for that option exceeds fifty percent of the Total Number of Votes Cast.
\asubsection{Two-Thirds}
In a Two-thirds Vote, a Quorum is reached if the Total Number of Votes Cast is equal to or exceeds two-thirds the Total Number of Possible Votes.
An option in the vote passes if the number of votes cast for the option equals or exceeds two-thirds of the Total Number of Votes Cast.
\asubsection{Three Tiered}
When voting on a set of three choices where each choice can be ranked in a definite order, a selection for either the lowest or highest option will only pass if the votes cast for that selection exceed fifty percent of the Total Number of Votes Cast.
If neither the highest or lowest selection exceeds fifty percent, the middle option will then automatically be passed.
In a Three-Tiered Vote, a Quorum is reached if the Total Number of Votes Cast is equal to or exceeds two-thirds of the Total Number of Possible Votes.


\asection{Ties Between Vote Options}
\asubsection{With Pass/Fail}
If the number of votes cast for the pass option equals the number of votes cast for the fail option, then the vote has failed.
\asubsection{With Multiple Options}
If multiple options may pass, a tie does not present a problem.
If only one option may pass, then the vote must be recast or tabled at the discretion of the current President of the Open Works Collective.
In the event of a tie in an Executive Board vote, the Chairperson may cast the tie-breaking vote.

% ARTICLE IV - AMENDING THIS CONSTITUTION
\article{Amending this Constitution}

\asection{Members}
The House Charter is a document drafted by the department of Residence Life, setting down the guidelines within which this House operates and from which it derives its authority.

\asection{Officers}
The House Constitution is written and maintained by the House and defines the major aspects, goals, and governing structure of the house.
It is reviewed annually by the Department of Residence Life.

\asubsection{Non-Semantic Changes}
There are two methods for non-semantic change to the Constitution.
A Maintainer may approve any proposed change that does not affect the meaning of the document.
Alternatively, the change may be presented to the House for discussion followed by an Immediate Vote.
A quorum of fifty percent of the Total Number of Possible Votes is required for passage.

\asubsection{Semantic Changes}
Any semantic change to the Constitution requires the change to be proposed in writing for discussion at a House Meeting.
Any modifications made due to the discussion are added to the written proposal and the modified proposal is posted in the House during the week.
The final proposal is presented the following week and ballots are distributed for a ballot House vote as described in \ref{Balloted Vote}.
The ballots are collected for a minimum of a forty-eight hour period.
A quorum of two-thirds of the Total Number of Possible Votes must cast ballots for the vote to be official.
A vote equaling or exceeding two-thirds of the number of votes cast is required for the change to be placed into the constitution.
The Constitution may be overridden by an immediate House vote as described in \ref{Immediate Vote}.
There must be a quorum of eighty-five percent of the Total Number of Possible Votes for the vote to be official.
A vote equaling or exceeding ninety percent of the number of votes cast is required for the override to take effect.

\asection{Constitution maintainers}

\asubsection{Maintainer Qualifications}
Maintainers must be Active or Alumni in good standing.

\asubsection{Maintainer Expectations}
Maintainers are expected to:
\begin{itemize}
	\item Review changes to the Constitution for grammar, spelling, and internal consistency
	\item Keep a public record of changes to the Constitution
	\item Participate in discussion of proposals
	\item Facilitate the creation of new changes to the Constitution by other members
	\item Be knowledgeable about the Constitution
\end{itemize}
Failure to meet any of these expectations is grounds for revocation of Maintainer status by the Executive Board.

\asubsection{Maintainer Selection}
Any member may nominate a qualified member for Maintainer status to the Executive Board for consideration.
The Executive Board may choose to approve or reject the nomination by Simple Majority vote.


%
% Could omit? - mark
%

%
% ARTICLE V - Financial Structure
%
\article{Officers}
The Executive Board is the main governing body of the House.
Its purpose is to provide leadership and direction for the House, to oversee the day-to-day operations of the House, and to initiate and organize programs and projects for the House.
It is composed of the seven permanent directors and the Chairperson.
\\*\\*
There is one permanent directorship for each major aspect of the government of the House and each one is chaired by an Executive Board member.
Ad Hoc directorships are created on an as-needed basis.
They are generally very task oriented and are chaired by a House member.
A directorship has some jurisdiction in its area of interest and often is responsible for the day-to-day decisions regarding its area of interest.
Any large expenditures or large effect decisions must be brought before the entire House
\asection{Members of the Executive Board}
\asubsection{Voting Members}
\begin{itemize}
    \item Outreach Director(s)
    \item  - Community
    \item  - Academic
    \item  - Industry
	\item Evaluations Director
	\item Social Director(s)
	\item Financial Director
	\item Research and Development Director(s)
	\item House Improvements Director
	\item Operational Communications Director
	\item House History Director
\end{itemize}
\asubsection{Non-Voting Members}
\begin{itemize}
	\item Chairperson
	\item House Secretary
	\item Ad Hoc Director(s)
\end{itemize}
\asection{Closed Executive Board}
Closed Executive Board Meetings are open only to the Chairperson, Voting Members of the Executive Board, and those with the express permission of the Executive Board.
A closed Executive Board meeting may be called at any time by any member of the Executive Board.
However, the Chairperson and at least two-thirds of the Voting Members must be present for the meeting to be called.

\asection{Responsibilities}
\renewcommand{\theenumi}{\alph{enumi}} % For this section, we want items to use letters
\asubsection{Responsibilities of the Executive Board}
\begin{enumerate}
	\item To hold a weekly meeting specific to their responsibilities and submit notes to the House
	\item To meet, as an Executive Board, at least weekly during the Standard Operating Session, as defined in \ref{Standard Operating Session} to discuss and report the operations of the House
	\item To report pertinent information to House members at the following House Meeting
	\item To maintain records of the goals defined by each previous Executive Board
	\item To act as a Judicial Board as defined in \ref{Judicial}
	\item To review major projects, as defined in \ref{Expectations of House Members}, presented to them by the Evaluations director
	\item To make the final vote regarding conditionals and appeals as defined in \ref{Membership Evaluation}
	\item To respect the privacy of House members confiding in the Executive Board, barring situations related to endangerment of oneself or others, sexual assault, or in the case of a Judicial Board
	\item To publish a document at the end of each semester to all Members stating House’s accomplishments of that semester
	\item To review and update the Constitution at the end of each Standard Operating Session, as defined in \ref{Standard Operating Session}.
		The constitution should remain up to date with current practices.
\end{enumerate}

\asubsection{Responsibilities of the Chairperson}
\begin{enumerate}
	\item To preside over Executive Board and House Meetings
	\item To exercise general supervision over the operations of the Executive Board
	\item To exercise general supervision over regular House activities
	\item To act as a liaison to the academic and administrative departments at RIT
	\item To act as a part of a Judicial Board as defined in \ref{Judicial}
	\item To cast tie-breaking vote in a split decision in an Executive Board vote
\end{enumerate}

\asubsection{Responsibilities of the Evaluations Director}
\begin{enumerate}
	\item To preside over Evaluations Meetings
	\item To exercise general supervision over Evaluations operations
	\item To oversee the screening, interviewing, and acceptance or rejection of prospective House members
	\item To oversee the Semi-Annual Evaluations of current House members
	\item To collaborate with the Residence Life Advisor to determine room change selection and any changes of membership residency
	\item To act as a part of a Judicial Board as defined in \ref{Judicial}
	\item To prepare the House for Open Houses, tours, and special events
\end{enumerate}

\asubsection{Responsibilities of the Social Director}
\begin{enumerate}
	\item To preside over Social Meetings in which House members are encouraged to bring ideas for social events
	\item To exercise general supervision over Social operations
	\item To oversee the organization, initiation, and execution of House activities and events
	\item To ensure that there is a variety of activities for House members to participate in throughout the academic year
\end{enumerate}
\article{Financial Structure}
\asection{Income}
\asubsection{Dues}
\asubsection{Donations}
\end{document}
